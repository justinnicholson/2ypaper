%Delcare document type
\documentclass[]{article}

%Article information
\author{Justin Nicholson}
\date{\today}
\title{Adjudicating Adjudication: a rational look}

%Call nessecary external packages
\usepackage[margin=1in]{geometry}
\usepackage{setspace}
\usepackage{fancyhdr, amsmath, amssymb}
\usepackage{graphicx}
\usepackage{setspace}
\usepackage{hyperref}
\pagestyle{fancy}
\usepackage{spverbatim}
\usepackage{placeins}
\usepackage{amsthm}
%Fancy Header options
\lhead{Draft Paper}
\rhead{Nicholson}
\headheight = 14pt
\doublespace
\begin{document}
\section{introduction}
The budget for the World Trade Organization totaled roughly \$217,504,702.00 in 2013,the most recent year on record. There are currently 149 cases in consultations, including 4 in limbo since 1995. a further 43 cases are awaiting the composition or decision of a panel. This includes one case filed by the US against Argentina in 1999 over footwear and a Canadian case against the EEC over duties on imports of cereals. 160 states are members or observers, which accounts for roughly 82 percent of extant states. Adjudication, not to mention accession, is costly, often slow and bears no legal status beyond what the member states attribute to it. The question then is, why bother?  \\

In a world of complete information, we would expect bilateral bargaining to provide a resolution that is strictly Pareto superior to any resolution reached through a costly mechanism. Although Fearon discusses war, the logic goes through for any model where the alternative imposes a cost. Fairly standard reasoning then, would lead us to conclude that the the dispute settlement mechanism must impart some value over standard bargaining. Theories of adjudication often lead from the observation that democratic states utilize the mechanism more often than nondemocratic states. This has been explained through various incentives that democratic institutions create. Specifically, the three broad schools of thought conclude that the pressures of divided government, the incentives created by open elections or the values inherent in democratic systems are ultimately responsible for this empirical regularity. \\
 
\section{contribution}
This paper makes two broad contributions to the substantive literature on WTO adjudication by addressing two issues related by the presence of non-random action. The first issue is the data available for quantitative analysis. In her analysis of the democratic propensity to adjudicate, Christina Davis utilizes count data to model the propensity to adjudicate for WTO members. While useful, this exhibits a problematic yet subtle issue.  For $Y_{it}\geq 1$ observations in the data accurately reflect adjudicated cases, which necessarily implies a prima facie trade dispute exists. However, $Y_{it} = 0$ is a thornier issue. In fact in this case, it is not well defined. A zero may indicate that there were no potential trade disputes to adjudicate in a dyad year, or it may indicate that no dispute was selected for adjudication, however these are not equivalent. One possible solution is the use of a Zero-Inflated Poisson model (ZIP), however with deeper data, we can directly study the initiation of adjudication from a set of \textit{potential} WTO disputes directly. \\

In part, event count models, and in particular ZIP/ZINB models mentioned above are abandoned because of the novel dataset constructed. The other reason directly relates to the characteristic primacy of strategic interaction in the nature of international politics. Making use of the dispute settlement mechanism is not an automatic process that varies stochastically with exogenous regressors. The decision is made by an individual \footnote{While these decisions are often made by groups, the problem becomes almost completely intractable when viewed this way. Decisionmaking bodies and responsibilities vary wildly across states, most internal documents are not easily accessable etc. For the purposes of this paper, we will impose the unitary actor assumption standard in IR literature.}  in response to a choice problem. If this were a choice problem solved in a vacuum, then standard statistical models might still apply. In this case however, it is almost certainly the case that actions are chosen based on the expected probabilities over other players actions. In short, this is a problem of strategic action. Signorino and Yilmaz show that using a standard statistical model when strategic interaction is present is equivalent to accepting omitted variable bias " where the omitted variables are nonlinear higher-order terms associated with expected utility calculations" \cite{SY2003}. Using a non-strategic model can lead the analyst to directly opposite inferences than their strategic brethren when strategic interaction occurs. \\

Bearing this warning in mind, I offer an analysis utilizing strategic backwards induction on an improved dataset which contains the universe of potential disputes for a certain type of temporary trade barrier. Each observation is a potential Anti-Dumping trade dispute, whether or not it became a WTO dispute. The combination of data and method should allow for more precise results regarding the determinants of strategic interaction at the WTO. \\

 The first stage of the SBI model examines the determinants of panel formation once a dispute is filed. Keeping in mind that nearly 90\% of panel rulings favor the plaintiff, expected probabilities are derived. These probabilities are then included in a model of adjudication initiation with surprising and novel results. Both the model and the results will be discussed in more detail later in the article. 

\subsection{data}
Any quantitative analysis of the WTO dispute settlement mechanism will face some idiosyncratic challenges. The data ranks first and foremost among these issues.   The observation that there are more WTO disputes filed by democracies, does not necessarily imply that WTO violations against democracies are more likely to be brought to the WTO. 
\subsubsection{identifying potential cases}
One  of the signal issues in conducting this analysis is identifying the universe of cases that have the potential to become a WTO dispute. It is not immediately obvious what a potential case might be in general. However, We can narrow down the potential options by examining the subset of WTO cases that arise out of dumping / anti-dumping disputes. WTO rules allow a members state to identify and preliminarily address claims of market dumping. An investigation can result in either an anti-dumping action being taken over the potential trade barrier, or no action taken. The state targeted with the anti-dumping measure can then decide whether or not to escalate the dispute to a WTO adjudication case. If escalated, this case can be either be settled in the consultation phase, or continue on to panel formation. Panel formational overwhelming results in a ruling for the plaintiff. \\

The kernel of the dataset is derived from The Temporary Trade Barrier dataset, compiled by Chad Bowen. This dataset takes an investigation into a dumping violation as an observation. I take a potential WTO case as any observation that results in at least a preliminary anti-dumping action logged in the anti-dumping portion of The Temporary Trade Barrier Database, where both the investigated and the investigating states are WTO members. \cite{BADD}. This results in 1696 observations, from 1993 through 2012\footnote{All cases considered in this paper from the GATT period were adjudicated after the formation of the WTO.}, although several observations are subsequently dropped in the analysis section due to lack of information. Although there is good reason to assume that the missing data might not be random, there are so few observations compared to the rest of the data, that it is inconsequential. Further, some observations are not unique; i.e. some anti-dumping measures result in more than one WTO dispute. While examining what factor might lead to a multiple-filing at the WTO, we treat these cases as the same as single-filings. non-unique records are eliminated. 

\subsubsection{more about the data}
A WTO dispute is a dyadic relationship involving two parties and a time period. One state is the ``plaintiff" or sender, whilst the other state in the pair is the ``defendant" or target. The WTO does not actively enforce it`s rules, instead it relies on a ``fire alarm" type enforcement mechanism. States need to file disputes, and we take the result as a binary choice, i.e. 0 or 1. For the purposes of this analysis, third party interest in a case is ignored.  \\

Most problematic is the sparsity of data on what change, if any, triggered the investigation and similarly what change in anti-dumping measures, if any, triggered the WTO dispute. Although it might seem forced by the data, treating investigations as largely exogenous however is not as problematic as it might appear. This is a decision often undertaken by an autonomous or quasi-autonomous government division, and in regards to a relatively technical article of the WTO charter and it can be argued that there are easier articles to manipulate if a trumped up challenge is desired.  Although there are often ``selection problems all the way down" in International Relations, the new data provides a deeper, if narrower, picture of dispute initiation than most other datasets available. 

\subsubsection{a further caveat - naming issues}
The original dataset was often not consistent in the naming of states. For example, take the state formerly known as the Union of Soviet Socialist Republics. Depending on the time period, it is sometimes coded as the USSR, Russia, The Russian Federation etc. Every effort was made to inspect and standardize inconsistent naming for our analysis.  Given the size of the dataset however, it is not clear that every single observation was sufficiently corrected. The appendix contains more detailed information, and in the few cases a coding decision might be controversial, an explanation of the logic behind the coding choices. \\

For now, note that continuity is stressed; i.e. if a case was filed against the USSR and continued by Russia after the fall of the Soviet Union, then the data was coded to allow analysis of the case against that entity, no matter its name. Similarly, I assume that the unified post-1991 state of Germany properly inherited the disputes initiated by and targeted against the entity formerly known as The Federal Republic of Germany (i.e. West Germany).

\subsection{the method}
The data is analyzed following 

\section{empirical findings}
\subsection{standard logit analysis}
\subsection{SBI analysis}
\section{conclusion}
\section{extensions}

\bibliography{2yp}
\bibliographystyle{plain}
\end{document}